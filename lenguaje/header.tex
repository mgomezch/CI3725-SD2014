\usepackage{newunicodechar}
\newunicodechar{η}{\ensuremath{\eta}}
\newunicodechar{λ}{\ensuremath{\lambda}}
\newunicodechar{∈}{\ensuremath{\in}}
\newunicodechar{∧}{\ensuremath{\land}}
\newunicodechar{∨}{\ensuremath{\lor}}
\newunicodechar{∩}{\ensuremath{\cap}}
\newunicodechar{∪}{\ensuremath{\cup}}
\newunicodechar{∷}{\ensuremath{::}}
\newunicodechar{≡}{\ensuremath{\equiv}}
\newunicodechar{𝔹}{\ensuremath{\mathbb{B}}}



\usepackage{polyglossia}
\setdefaultlanguage{spanish}
\hyphenation{do-cu-men-to}



\usepackage[margin=2.5cm]{geometry}



\usepackage{anyfontsize}
\usepackage{lastpage}

\makeatletter
\let\mytitle\@title
%\let\mysubtitle\@subtitle
\let\myauthor\@author
\let\mydate\@date
\makeatother

\usepackage{titling}
\title{\mytitle}
%\subtitle{\mysubtitle}
\author{\myauthor}
\date{\mydate}

\pretitle{
  \begin{center}
  \fontsize{50}{60}
  \selectfont
  \sc
}

\posttitle{
  \end{center}
}

\usepackage[normalem]{ulem}

\renewcommand{\maketitlehookd}{
  \textbf{Este documento es solo una propuesta y no especifica requerimientos para ninguna evaluación.  No debe suponerse que alguna evaluación vaya a ser basada en él a menos que sea publicado oficialmente por los profesores a cargo del curso; en ese caso, la versión normativa del documento será la que se publique, que no contendrá esta nota, y esta versión será considerada inválida y no deberá ser usada como referencia para ningún propósito.}

  Este documento es un borrador, y se espera que tenga deficiencias de diseño, del redacción, orthografícas y d\emph{e} t\textbf{ipo}g\textsuperscript{r}a\sout{f}\textsubscript{ía}.
}



\usepackage{fancyhdr}
\usepackage{calc}

\pagestyle{fancyplain}

\def\headerlogo{\includegraphics[width=75pt]{header.png}}
\addtolength{\headheight}{\heightof{\headerlogo}}
\addtolength{\textheight}{0pt-\heightof{\headerlogo}}
%\addtolength{\headheight}{\baselineskip}

\fancyhead{}
\renewcommand{\headrulewidth}{0pt}
\fancyhead[R]{
  \parbox[b]{0.7\textwidth}{
    \begin{flushright}
      \scriptsize{
        \thetitle \\
        \theauthor
      }
    \end{flushright}
    \vspace{-14pt}
  }
}

\fancyhead[L]{
  \headerlogo
}


\fancyfoot{}
\fancyfoot[L]{
  \parbox[t]{0.9\textwidth}{
    \scriptsize{
      CI3725 (Traductores e intérpretes) — Septiembre–Diciembre 2014 — Universidad Simón Bolívar
    }
  }
}
\fancyfoot[R]{\parbox[t]{0.02\textwidth}{\textbf{\thepage}}}

\hypersetup{
  xetex,
  pdftitle={El lenguaje de programación FOO},
  pdfsubject={ 2 de CI3825 (Enero–Marzo 2012, USB)},
  pdfauthor={David Lilue, Ernesto Hernández-Novich, Karen Troiano, Matteo Ferrando, Manuel Gómez},
  pdfcreator={Manuel Gómez},
  pdfkeywords={CI3725},
  pdflang={es-VE},
  bookmarks=true,
  bookmarksnumbered=true,
  pdfpagelabels=true,
  pdfpagemode=UseOutlines,
  pdfstartview=FitH,
  linktocpage=true,
  colorlinks=true,
  linkcolor=blue,
  plainpages=true,
  setpagesize=false
}



\urlstyle{tt}
